% !TeX root = RJwrapper.tex
\title{true}
\author{by Your R. Name}

\maketitle

\abstract{
The abstract of your thesis.
}

\tableofcontents

\chapter*{Introduction}

Welcome to the \LaTeX~thesis template. If you've never used \TeX~or
\LaTeX~before, you'll have an initial learning period to go through, but
the results of a nicely formatted thesis are worth it for more than the
aesthetic benefit: markup like \LaTeX~is more consistent than the output
of a word processor, much less prone to corruption or crashing and the
resulting file is smaller than a Word file. While you may have never had
problems using Word in the past, your thesis is going to be about twice
as large and complex as anything you've written before, taxing Word's
capabilities. If you're still on the fence about using \LaTeX, read the
Introduction to LaTeX on the CUS site as well as skim the following
template and give it a few weeks. Pretty soon all the markup gibberish
will become second nature.

\section{Why use it?}

\LaTeX~does a great job of formatting tables and paragraphs. Its
line-breaking algorithm was the subject of a PhD.\thinspace thesis. It
does a fine job of automatically inserting ligatures, and to top it all
off it is the only way to typeset good-looking mathematics.

\section{Who should use it?}

Anyone who needs to use math, tables, a lot of figures, complex
cross-references, IPA or who just cares about the final appearance of
their document should use \LaTeX. At Reed, math majors are required to
use it, most physics majors will want to use it, and many other science
majors may want it also.

\chapter{R Markdown}

This is an R Markdown document. Markdown is a simple formatting syntax
for authoring HTML, PDF, and MS Word documents. For more details on
using R Markdown see \url{http://rmarkdown.rstudio.com}.

When you click the \textbf{Knit} button a document will be generated
that includes both content as well as the output of any embedded R code
chunks within the document. You can embed an R code chunk like this:

\begin{Schunk}
\begin{Sinput}
summary(cars)
\end{Sinput}
\begin{Soutput}
#>      speed           dist       
#>  Min.   : 4.0   Min.   :  2.00  
#>  1st Qu.:12.0   1st Qu.: 26.00  
#>  Median :15.0   Median : 36.00  
#>  Mean   :15.4   Mean   : 42.98  
#>  3rd Qu.:19.0   3rd Qu.: 56.00  
#>  Max.   :25.0   Max.   :120.00
\end{Soutput}
\end{Schunk}

\address{
Your R. Name\\
\\
\\
}


