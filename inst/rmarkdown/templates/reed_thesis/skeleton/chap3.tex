\documentclass[]{article}
\usepackage{lmodern}
\usepackage{amssymb,amsmath}
\usepackage{ifxetex,ifluatex}
\usepackage{fixltx2e} % provides \textsubscript
\ifnum 0\ifxetex 1\fi\ifluatex 1\fi=0 % if pdftex
  \usepackage[T1]{fontenc}
  \usepackage[utf8]{inputenc}
\else % if luatex or xelatex
  \ifxetex
    \usepackage{mathspec}
    \usepackage{xltxtra,xunicode}
  \else
    \usepackage{fontspec}
  \fi
  \defaultfontfeatures{Mapping=tex-text,Scale=MatchLowercase}
  \newcommand{\euro}{€}
\fi
% use upquote if available, for straight quotes in verbatim environments
\IfFileExists{upquote.sty}{\usepackage{upquote}}{}
% use microtype if available
\IfFileExists{microtype.sty}{%
\usepackage{microtype}
\UseMicrotypeSet[protrusion]{basicmath} % disable protrusion for tt fonts
}{}
\usepackage[margin=1in]{geometry}
\ifxetex
  \usepackage[setpagesize=false, % page size defined by xetex
              unicode=false, % unicode breaks when used with xetex
              xetex]{hyperref}
\else
  \usepackage[unicode=true]{hyperref}
\fi
\hypersetup{breaklinks=true,
            bookmarks=true,
            pdfauthor={},
            pdftitle={},
            colorlinks=true,
            citecolor=blue,
            urlcolor=blue,
            linkcolor=magenta,
            pdfborder={0 0 0}}
\urlstyle{same}  % don't use monospace font for urls
\setlength{\parindent}{0pt}
\setlength{\parskip}{6pt plus 2pt minus 1pt}
\setlength{\emergencystretch}{3em}  % prevent overfull lines
\providecommand{\tightlist}{%
  \setlength{\itemsep}{0pt}\setlength{\parskip}{0pt}}
\setcounter{secnumdepth}{0}

%%% Use protect on footnotes to avoid problems with footnotes in titles
\let\rmarkdownfootnote\footnote%
\def\footnote{\protect\rmarkdownfootnote}

%%% Change title format to be more compact
\usepackage{titling}

% Create subtitle command for use in maketitle
\newcommand{\subtitle}[1]{
  \posttitle{
    \begin{center}\large#1\end{center}
    }
}

\setlength{\droptitle}{-2em}
  \title{}
  \pretitle{\vspace{\droptitle}}
  \posttitle{}
  \author{}
  \preauthor{}\postauthor{}
  \date{}
  \predate{}\postdate{}


% Redefines (sub)paragraphs to behave more like sections
\ifx\paragraph\undefined\else
\let\oldparagraph\paragraph
\renewcommand{\paragraph}[1]{\oldparagraph{#1}\mbox{}}
\fi
\ifx\subparagraph\undefined\else
\let\oldsubparagraph\subparagraph
\renewcommand{\subparagraph}[1]{\oldsubparagraph{#1}\mbox{}}
\fi

\begin{document}
\maketitle

\section{Mathematics and Science}\label{math-sci}

\subsection{Math}\label{math}

\TeX~is the best way to typeset mathematics. Donald Knuth designed
\TeX~when he got frustrated at how long it was taking the typesetters to
finish his book, which contained a lot of mathematics. One nice feature
of R Markdown is its ability to read \LaTeX~code directly.

If you are doing a thesis that will involve lots of math, you will want
to read the following section which has been commented out. If you're
not going to use math, skip over this next commented section.

\subsection{Chemistry 101: Symbols}\label{chemistry-101-symbols}

Chemical formulas will look best if they are not italicized. Get around
math mode's automatic italicizing by using the argument
\texttt{\$\textbackslash{}mathrm\{formula\ here\}\$}, with your formula
inside the curly brackets.

So, \(\mathrm{Fe_2^{2+}Cr_2O_4}\) is written
\texttt{\$\textbackslash{}mathrm\{Fe\_2\^{}\{2+\}Cr\_2O\_4\}\$}

\noindent Exponent or Superscript: O\(^{-}\)

\noindent Subscript: CH\(_{4}\)

To stack numbers or letters as in \(\mathrm{Fe_2^{2+}}\), the subscript
is defined first, and then the superscript is defined.

\noindent Angstrom: \{\AA\}

\noindent Bullet: CuCl \(\bullet\) 7H\({_2}\)O

\noindent Double Dagger: \ddag

\noindent Delta: \(\Delta\)

\noindent Reaction Arrows: \(\longrightarrow\) or
\(\xrightarrow{solution}\)

\noindent Resonance Arrows: \(\leftrightarrow\)

\noindent Reversible Reaction Arrows: \(\rightleftharpoons\) or
\(\xrightleftharpoons[ ]{solution}\) (the latter requires the
\texttt{chemarr} \LaTeX~package which is automatically loaded in this
template)

\subsubsection{Typesetting reactions}\label{typesetting-reactions}

You may wish to put your reaction in a figure environment, which means
that LaTeX will place the reaction where it fits and you can have a
figure legend if desired:

\begin{figure}[htbp]
\begin{center}
$\mathrm{C_6H_{12}O_6  + 6O_2} \longrightarrow \mathrm{6CO_2 + 6H_2O}$
\caption{Combustion of glucose}
\label{combustion of glucose}
\end{center}
\end{figure}

\subsection{Other examples of reactions}

\(\mathrm{NH_4Cl_{(s)}} \rightleftharpoons \mathrm{NH_{3(g)}+HCl_{(g)}}\)

\noindent \(\mathrm{MeCH_2Br + Mg} \xrightarrow[below]{above} \mathrm{MeCH_2\bullet Mg \bullet Br}\)

\subsection{Physics}\label{physics}

Many of the symbols you will need can be found on the math page
\url{http://web.reed.edu/cis/help/latex/math.html} and the Comprehensive
\LaTeX~Symbol Guide
(\url{http://mirror.utexas.edu/ctan/info/symbols/comprehensive/symbols-letter.pdf}).

\section{Biology}

You will probably find the resources at
\url{http://www.lecb.ncifcrf.gov/~toms/latex.html} helpful, particularly
the links to bsts for various journals. You may also be interested in
TeXShade for nucleotide typesetting
(\url{http://homepages.uni-tuebingen.de/beitz/txe.html}). Be sure to
read the proceeding chapter on graphics and tables.

\end{document}
